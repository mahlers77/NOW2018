% Please make sure you insert your
% data according to the instructions in PoSauthmanual.pdf
\documentclass{PoS}

\title{Multimessenger Astrophysics : Session Summary}

\ShortTitle{Short Title for header}

\author{Markus Ahlers\\
        Niels Bohr International Academy \& Discovery Centre, Niels Bohr Institute,\\University of Copenhagen, Blegdamsvej 17, DK-2100 Copenhagen, Denmark\\
        E-mail: \email{markus.ahlers@nbi.ku.dk}}

\author{Manuela Vecchi\\
       KVI-Center for Advanced Radiation Technology,\\University of Groningen, NL-9700 AB Groningen, Netherlands\\Instituto de F\'isica de Sao Carlos,\\ Universidade de Sao Paulo, CP 369, 13560-970, Sao Carlos, Sao Paulo, SP, Brazil\\
        E-mail: \email{m.vecchi@rug.nl}}

\abstract{This will the place for the abstract of our proceedings\
          ...........................}

\FullConference{Neutrino Oscillation Workshop (NOW2018)\\
		9 - 16 September, 2018\\
		Rosa Marina (Ostuni, Brindisi, Italy)}

\begin{document}

\section{Introduction {\it (Manuela \& Markus)}}

Introduction and highlights, {\it e.g.} TXS 0506+56~\cite{IceCube:2018cha,IceCube:2018dnn}

\section{Astrophysical Neutrinos {\it (Markus)}}

\begin{itemize}
\item Marek Kowalski - {\it IceCube-Gen2 and IceCube-Upgrade}~\cite{Aartsen:2014njl}
\item Maurizio Spurio - {\it Overview of ANTARES neutrino telescope: multimessenger results }
\item Rosa Coniglione - {\it First results and perspectives of the KM3NeT/ARCA Detector}~\cite{Adrian-Martinez:2016fdl}
\end{itemize}

\section{Supernova Neutrinos {\it (Markus)}}

\begin{itemize}
\item Francesco Capozzi - {\it Distinguishing SN $\nu$ flavour equalization from a pure MSW effect}
\item Rasmus Hansen - {\it Extended Neutrino Sphere Effects on SN $\nu$ Oscillations}
\item Giulia Pagliaroli - {\it Combined search of MeV $\nu$s and GWs from astro sources}
\end{itemize}

\section{Gamma Rays {\it (Manuela)}}
\textcolor{red}{Hello Markus, I am just writing down some notes  don't worry if it looks too long ;)}

\begin{itemize}
\item Elisabetta Bissaldi - {\it Multimessenger studies with the Fermi satellite}

Elisabetta Bissaldi reports about multimessenger studies with the \textit{Fermi} gamma ray satellite~\cite{Bissaldi}, operating in space since 10 years. The \textit{Fermi} payload is composed by the Gamma Ray Burst Monitor (GBM),  and the  Large Area Telescope (LAT). The two instruments have complementary science goals: the former was designed to study Gamma Ray Bursts (GRBs) by detecting gamma 
rays between 8 keV and 40 MeV, observing the entire unoccupied sky with absolute timing of 2 $\mu$s, while  the latter is devoted to the higher energies, from 20 MeV to 300 GeV and observes the entire sky every 3 hours with absolute timing of about 300  ns. 
%Since 2008, the Fermi GBM detected over 2000 GRBs, while the LAT detected almost 150 GRBs. 
The Fermi satellite is a key instrument to perform multi-messenger astrophysics, giving the high pointing accuracy, as well as the high quality data for both temporal and spectral analyses.  Moreover, the transient sources can be searched for by means of several trigger strategies.
The  electromagnetic (EM) counterpart of a gravitational wave generated by the merger of two neutron stars, was unambiguously detected for the first time on August 17th 2017 by the Fermi GBM~\cite{Gold2017}~\cite{Abbott2017}: the event was found to take place in the galaxy NGC 4993, located 44 Mpc away. %The GW was actually detected X second after the GBM detection: this delay being in agreement with general relativity. 
 

The detected EM emission includes three main components~\cite{TheMMpaper}: (i) a prompt, short GRB, demonstrating the association between GW and GRBs (ii) the first observation in ultraviolet, optical and infrared detection of a  kilonova, an object  due to the radioactive decay of heavy elements formed by neutron capture, followed by (iii) the delayed X-rays and radio counterparts. 
No coincident, high-energy neutrino was detected  from the area of GW170817. Extended search for neutrinos in the direction of the source was carried out for two weeks following the merger, but it found no significant neutrino emission. 

A new milestone in the frame of multimessenger astrophysics was achieved a few weeks later, in September 22nd 2017, when a  neutrino event of about 290 TeV detected by IceCube was found to be coincident in direction and time with a gamma-ray flare from the blazar TXS 0506+056~\cite{IceCube:2018dnn}.  This event happened in a period of flaring activity in GeV gamma rays detected by \textit{Fermi}-LAT and 
 above 100 GeV by the Major Atmospheric Gamma-ray Imaging Cherenkov (MAGIC) telescope~\cite{IceCube:2018cha}~\cite{MAGIC}. 

The probability of a high-energy neutrino being detected by chance coincidence with a flaring blazar from Fermi-LAT catalogs was found to be disfavored at a 3$\sigma$ confidence level. Ice Cube archival searches  found a compelling evidence for a three months ``neutrino flare'' in 2014-2015 from the blazar, however no gamma ray  counterpart was found in Fermi-LAT data, the neutrino luminosity being five times higher than the corresponding gamma-ray luminosity.

While the detection of the high energy neutrino alone indicates acceleration of protons in the jet of the blazar to energies above the CR \textit{knee} (10$^{15}~eV$), the combination with the EM data allow us to probe the maximum energy that they can attain.

\item Andrew Taylor reports about {\it Multimessenger observations with IACTs}~\cite{Taylor}.

The combined electron plus positron spectrum was measured by the High-Energy Stereoscopic System (H.E.S.S.), located in Namibia, from 250 GeV to 20 TeV. This measurement can be described by a smooth broken power law, with no sharp features. This result is extremely important, since it allow us to exclude models that describe prominent features from nearby sources such as Vela.

The search for a TeV gamma ray counterpart of the GW170817 event allowed setting upper limits from HESS and other IACTs~\cite{TheMMpaper}.


The prompt diffusion of the sky position of the IceCube-170922A alert triggered an extensive multiwavelength (MWL) campaign by many telescopes on ground and in space, from radio frequencies up to VHE gamma-rays. In the VHE band, TXS 0506+056 was followed up by several Imaging Atmospheric Cherenkov Telescopes (IACTs), with the earliest one starting a few hours after the neutrino alert. In particular,  1.3 hours of observations in the direction of the blazar TXS 0506+056 were performed using the HESS experiment about 4 hours after the circulation of the neutrino alert. Additional observations on subsequent nights were also carried out, but no clear signal was found. Upper limits at 95$\%$ CL on the gamma-ray flux were derived accordingly 
$ \Phi < 7.5 \times 10^{-12} cm^{-2} s^{-1} $ 
during the H.E.S.S. observation period. 


The production of high-energy neutrinos in astrophysical environments is expected to occur mainly through the decay of charged pions produced in inelastic collisions between high energy protons and ambient target particles, which can be either matter (pp interactions) or low-energy photons (pγ interactions). The present multi-messenger studies are able to constrain this scenarios and shed light on the neutrino and cosmic ray origin. 
%In AGN jets that typically consist of low-density plasma, the latter is generally the favored channel. 
%The simplest scenario to explain the neutrino emission from blazars is the one-zone lepto-hadronic model, where a magnetized compact region inside the relativistic jets carries a population of relativistic electrons and protons. The production of high-energy neutrinos in astrophysical environments is expected to occur mainly through the decay of charged pions produced in inelastic collisions between high-energy protons and ambient target, while synchrotron-pair cascades of secondary particles are responsible for the high energy part of the electromagnetic emission.


Preliminary HESS results of the Centaurus A galaxy~\cite{Taylor}  show the first extra-galactic extended source in the TeV gamma rays.

\item Gerd Kunde reports about {\it HAWC and Multimessenger Physics}~\cite{Kunde}. 
The High Altitude Water Cherenkov detector is
located in Mexico at an altitude of 4100 meters and
is used to sample the charged particles created in the air showers generated by TeV gamma rays passing through the Earth atmosphere. It is sensitive to the energy range 0.5-100 TeV and
during each 24 hour period it observes two-thirds of the sky. 

Multi-wavelength and multi-messenger observations are 
essential to understanding the gamma ray sky. The HAWC 
experiment is sharing data with (and following up triggers from) many astroparticle physics experiments, including \textit{Fermi}-LAT, IceCube and Virgo-LIGO. 
HAWC performed the search for TeV gamma rays in association with the binary neutron star coalescence candidate (later designated GW170817) detected by the Virgo and LIGO gravitational wave detectors~\cite{TheMMpaper}. At the time of the trigger the region was not visible to HAWC, so that no prompt follow-up was possible. About 9 hours after the trigger, the region became to be visible and was observed for 2 hours, allowing setting 95$\%$ confidence level upper limits for energies above 1 TeV, assuming a -2.5 spectrum. 

Concerning the neutrino event reported by the IceCube Collaboration, whose multimessenger observation was reported in~\cite{IceCube:2018cha}, HAWC found no $\gamma$-ray source above 1 TeV at the location of TXS 0506+056, either close to the time of the neutrino alert or in archival data taken since November 2014.

The recent detection of extended~TeV~gamma-ray emission coincident with the 
locations of two nearby middle-aged pulsars (Geminga and PSR 
B0656+14)~\cite{HAWCpositrons} recently triggered a lot of interest. The HAWC observation of the spectral and spatial properties of these sources are used
to constrain their contribution to the positron flux at Earth, measured by AMS-02~\cite{positrons}. Assuming that all the observed gamma-ray emission at TeV energies is
produced by relativistic electron and positron pairs, the electron and positron
flux produced by these sources at Earth was computed. The measured TeV emission profile is used to constrain the diffusion of particles away from these
sources, and it is found to be much slower than previously assumed. The HAWC Collaboration thus concluded that, under the assumption of isotropic and homogeneous diffusion, the two pulsars are ruled out as sources of the CR positron flux.



%Gamma rays at these energies are mainly produced by 
%inverse Compton scattering of  positrons or electrons on
%lower-energy photons (CMB) as well as by by synchrotron radiation. 
%The diffusion coefficient increases
% is chosen to be 1/3, motivated by the Kolmogorov turbulence
%model.





\end{itemize}

\section{Cosmic Rays {\it (Manuela)}}
\textcolor{red}{Hello Markus, I am just writing down some notes  don't worry if it looks too long ;)}
\begin{itemize}
\item Fiorenza Donato - {\it Galactic cosmic ray anomalies}
reports about recent improvements in the field of galactic cosmic rays (CRs)~\cite{Donato}.
%, mainly in view of the latest data from the Alpha Magnetic Spectrometer (AMS-02). 
Following the hints provided by previous balloon experiments, indicating a transition in the spectral index of CR 
proton, helium and heavier nuclei, the 
PAMELA Collaboration published in 2011
precise measurements of proton and helium fluxes as a function of rigidity (the rigidity is given by the particle momentum over the charge $R=\frac{pc}{Ze}$)~\cite{PAMELApHe} between 1 GV to 1.2 TV, showing a clear feature above 200 GeV. 
The AMS-02 collaboration in 2015~\cite{proton}~\cite{Aguilar:2015ctt} showed  
that both the proton and helium spectra
 cannot be described as a single power law (between 1 GV to 1.8 TV), and that
a transition in the spectral index takes place above 200 GeV. 
% The single power law behaviour, namely
%$\Phi(R)~= C R^{-\gamma}$.
The transition in the spectral index  can be 
described as follows: 
\begin{equation}
\Phi(R)= C R^{-\gamma} \Big( 1 + \frac{R}{R_0}^{\Delta \gamma/s} \Big)^s
\end{equation}
where the parameter $s$ quantifies  the smoothness of the transition of the
spectral index, from $\gamma$ to $\gamma+\Delta \gamma$, that occurs at the rigidity $R_0$. The fit obtained in~\cite{proton} yields $\gamma \sim -2.814$
for a transition rigidity value $R_0 \sim 366 ~GV$ and $\Delta \gamma \sim 0.133$.

Moreover, not only the helium flux cannot be described by a single power law, but the ratio
between the proton and the helium flux is rigidity-dependent. This behavior is not expected, and it was also observed in the flux ratio of other species,
like C/p, O/p~\cite{heco}. 

Lithium, Beryllium and Boron are mostly secondary CR species,
produced by fragmentation of heavy nuclei (mainly C N O) propagating in the interstellar medium, made of protons and helium. 
The three fluxes~\cite{libeb} deviate from a single power law above 200 GV in an identical way.
The  secondary to primary flux ratios are very sensitive to the propagation of CRs in the Milky Way. In particular, the Boron to Carbon flux ratio (hereafter B/C), has been measured by a variety of CR experiments and
can be considered as \textit{standard candle} for the study of galactic CRs propagation. 

In order to reproduce the observed transition in the spectral index above 200 GeV, the transition can be generated  either at the source (injection or acceleration) 
either during the propagation in the Milky Way. The first indications in favor of a diffusive propagation origin for these spectral features was provided in~\cite{Genolini:2017dfb}, based on a semi-analytic approach for the transport equation. 
The observed hardening could be at least partially explained by a re-acceleration inside the sources of CRs, as discussed in~\cite{Blasi:2017caw}.
It is also important to mention another unexpected result: the fact that positrons, protons and antiprotons show identical rigidity dependence above 60 GV, while electrons have a different behavior~\cite{antip}. This results can be interpreted as a  coincidence or as the actual result of underlying physics. In the last case it would imply that both antiprotons and positrons are secondaries, implying small energy losses for these particles, in striking disagreement with the conventional scenario in which positrons undergo severe energy losses during their propagation, in view of the low mass (about 1/2000 times smaller than antiprotons).

While describing the current status of CR antiprotons, F. Donato also mentions two important points that emerge in view of the precision of AMS-02 data: the comparison between the expected antiproton flux from conventional astrophysical processes (whose \textit{baseline} is derived from the B/C analysis) and the precise AMS-02 data is still an open issue, especially because the antiproton production cross sections should be known with high accuracy in order to avoid introducing high theoretical uncertainties. 
In this context, the recent  LHCb data improve the experimental scenario. 

In view of the previous points, antiprotons data allow setting strong upper bounds on the dark matter annihilation cross section or to improve the fit with respect to the secondaries 
alone adding a tiny DM contribution. 
In other words, the possible room left for a primary component, likely due to dark matter (DM), is not yet clearly established, even though recent papers claim for a statistically significant excess with respect to the astrophysical background. 

CR electrons and positrons account for about $1\%$ of CRs. However, they carry important information about the
local universe, since the typical propagation length of TeV electrons and positrons is smaller than $\simeq 0.3 kpc$. 

Recent positrons measurements are not in agreement with the astrophysical background from conventional models, leaving the room for a primary component whose nature, either astrophysical or exotic, is still highly debated. At present, it cannot be excluded that nearby and
very old sources may contribute to the flux at Earth even if they are no longer visible at any
energy of the electromagnetic band. A strategy to constrain the source contributions of local
known sources is to model their multi-wavelength emission and to connect it to the emitted
CRs. A recent work~\cite{Manconi:2018azw} aims at  quantifying the contribution of local known
sources, in particular from two SNRs which are widely considered as the main candidates, namely Vela and Cygnus Loop, by combining three complementary measurements: (i) the radio flux from Vela and Cygnus Loop at all the available frequencies, (ii) the combined electron plus positron flux from five experiments from the GeV to tens of TeV energy, (iii) the electron plus positron dipole anisotropy upper limits from 50 GeV to about 1 TeV. 
The model is compatible with all the measurements, including the anisotropy bounds, providing strong arguments towards the understanding of the sources of CR electrons and positrons. 







\item Rogerio de Almeida - {\it Anisotropy Studies with the Pierre Auger Observatory}. 
Rogerio M. de Almeida~\cite{Rogerio} presents the latest results on anisotropies studies with the Pierre Auger Observatory (hereafter PAO). 
The Observatory~\cite{PAOdet} has been in successful operation since completion in 2008,  investigating the origin and characteristics of CRs above 10$^{17}$ eV. The PAO detects Ultra High Energy Cosmic Rays (UHECRs) through the extensive air showers that these particles induce in the atmosphere. The identification of CR~-~induced showers is performed by combining an array of  1660  water Cherenkov detectors, with a duty cycle of about 100$\%$, and 27 fluorescence detectors used during dark nights, with a 15$\%$ duty cycle. The combination of the two techniques provides an energy scale insensitive to assumptions on the primary masses and on simulations uncertainties, with systematic uncertainties in the energy scale of 14$\%$.

The latest measurements of the CR energy spectrum~\cite{PAOICRC2017}, obtained with the data collected from January 2004 until December
2016, show that the flux is well described by a broken power law plus a smooth suppression
at the highest energies. An ankle is found at
$E_{ankle} = (5.08 \pm 0.06(stat.) \pm 0.8(syst.)) \times 10^{18} eV$, while the suppression is at $E_{s} = (3.9 \pm 0.2(stat.) \pm
0.8(syst.)) \times 10^{19} eV$. 

The UHECRs composition measurements are performed  
by using the depth of the position of the maximum in the
number of shower particles, provided by the fluorescence detector. The composition measurement by PAO reveals a transition from light to mixed composition above $\simeq 10^{19} eV$. A recent analysis by the PAO~\cite{Aab:2018chp} provides evidence in the arrival direction of
UHECRs on an intermediate angular scale, indicative of excess arrivals from strong nearby sources. 
The arrival directions of 5514 events with energies above 39 EeV are compared
with two distinct populations of extra-galactic $\gamma$-ray emitters: active galactic nuclei from the second catalog of hard Fermi-LAT sources (2FHL) and starburst galaxies from a sample that was examined with \textit{Fermi}-LAT. The starbust model fits the data better than the hypothesis of isotropy, with statistical significance of $4\sigma$. However, caution is required when identifying these objects as the preferred 
sources prior to understanding the impact of magnetic deflections. 

Large scale anisotropies can be used to infer aspects of the global distribution of sources. A strong evidence (5.2$\sigma$) of dipolar anisotropy for CRs with energies above 8 EeV is reported~\cite{Aab:2017tyv}, while no significant second harmonic amplitude is observed. This results can be interpreted as a strong evidence against the galactic origin of UHECRs. The energy dependence of the amplitude and direction of the three-dimensional dipole are presented in~\cite{Aab:2018mmi} for events with energy higher than 4 EeV. The energy dependence is expected, because of the smaller deflection experienced
by UHECRs at higher rigidities as well as large attenuation suffered by UHECRs from distant sources.  No clear dependence on energy as found for the dipole direction, but regardless the energy, the dipole direction point towards a region outside our galaxy. 
Plausible explanations for the observed dipole-like structure
include the effect of the diffusive propagation as 
as well as the effect of the known anisotropy in the distribution of the sources. 
\item Martin Lemoine - {\it On ultra-high energy cosmic rays}
Martin Lemoine~\cite{Lemoine} summarizes the current understanding of UHECRs origin, in view of the most recent experimental findings. 
The main challenge in this field is currently to understand the observed UHECR spectra and mass composition and relating
them to potential sources, most likely extra-galactic~\cite{Aloisio}. 

The two largest and most precise detectors of UHECR currently in operation are the
Pierre Auger Observatory (PAO) in Argentina (Mendoza) and the Telescope Array in the
USA (Utah). Both detectors exploit the hybrid concept, combining an array of
surface detectors to sample extensive air showers when they reach the ground and
fluorescence telescopes, overlooking the surface array, to collect the fluorescence light of the
excited atmospheric nitrogen. 




As it was discussed in the talk by R. de Almeida~\cite{Rogerio}, the energy spectrum of UHECRs shows a clear suppression at $E_{s} = (3.9 \pm 0.2(stat.) \pm
0.8(syst.)) \times 10^{19} eV$, compatible with the expected value for the GZK cutoff, where the universe becomes opaque to protons. However, this spectral feature could also be indicating the maximal energy that can be achieved at acceleration sites. 
The UHECR composition and arrival direction constitute two additional observables to shed light on the problem. The PAO reports evidence for a mixed composition at high energies, resulting in an increasingly heavier composition above $10^{18} eV$. The results from the Telescope Array (TA), initially in tension with those from PAO, have been shown to be compatible with mixed composition which best describes the PAO data~\cite{vitor}. 

The PAO and TA, being located on two different hemispheres, have access to different regions of the sky: the TA reported the existence of a hot spot in the Northern sky with a significance of $3.4~\sigma$, while the PAO reports 
a strong evidence (5.2$\sigma$) of dipolar anisotropy for CRs with energies above 8 EeV~\cite{Aab:2017tyv}. 


The previously discussed experimental findings must be combined in order to shed light on the origin of UHECRs. 

The possibility to point back to the sources of UHECRs depends crucially on the composition, since  
at a given energy $E$, a nucleus of charge $Z$ is typically deflected by an angle $Z$ times larger than a proton, so that protons are expected to produce stronger anisotropies than nuclei. 
Moreover, secondary gamma rays and neutrinos~\cite{AlvesBatista} are generated in 
interactions whose energy threshold depends on the mass number $A$, implying that the flux of these particles also depend on the UHECR composition. In this frame, the pure proton composition is in tension with secondary diffuse backgrounds of TeV gamma rays and neutrinos. 



\end{itemize}

\section{Conclusions {\it (Manuela \& Markus)}}


\section*{Acknowledgements}
MA acknowledges support by Danmarks Grundforskningsfond (project no.~1041811001) and \textsc{Villum Fonden} (project no.~18994).

\bibliographystyle{JHEP}
\bibliography{references}


\end{document}
