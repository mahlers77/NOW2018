% Please make sure you insert your
% data according to the instructions in PoSauthmanual.pdf
\documentclass{PoS}

\usepackage{url}
 
\title{Multimessenger Astrophysics : Session Summary}

\ShortTitle{Short Title for header}

\author{Markus Ahlers\\
        Niels Bohr International Academy \& Discovery Centre, Niels Bohr Institute,\\University of Copenhagen, Blegdamsvej 17, DK-2100 Copenhagen, Denmark\\
        E-mail: \email{markus.ahlers@nbi.ku.dk}}

\author{Manuela Vecchi\\
       KVI-Center for Advanced Radiation Technology,\\University of Groningen, NL-9700 AB Groningen, Netherlands\\Instituto de F\'isica de Sao Carlos,\\ Universidade de Sao Paulo, CP 369, 13560-970, Sao Carlos, Sao Paulo, SP, Brazil\\
        E-mail: \email{m.vecchi@rug.nl}}

\abstract{We summarize the activities of the ``Multimessenger Astrophysics'' parallel session of the Neutrino Oscillation Workshop (NOW2018).}

\FullConference{Neutrino Oscillation Workshop (NOW2018)\\
		9 - 16 September, 2018\\
		Rosa Marina (Ostuni, Brindisi, Italy)}

\begin{document}

\section{Introduction {\it (Manuela \& Markus)}}

%Introduction and highlights, {\it e.g.} TXS 0506+56~\cite{IceCube:2018cha,IceCube:2018dnn}

The neutrino has unique properties that makes it an ideal messenger for astronomy. It is not deflected by Galactic and extra-galactic magnetic fields, as it happens for charged cosmic rays, and its arrival direction points back to its origin. It is also not affected by absorption in cosmic radiation backgrounds, as it is the case for $\gamma$-rays above TeV, and it therefore provides a probe of distant and therefore early cosmic sources. Its weak interaction with matter allows to study dense environments, like proto-neutron stars from the core-collapse of supernovae, that are otherwise not visible in photons.

High-energy neutrinos are a result of pion production in cosmic ray (CR) interactions with background photons and gas. Charged pion decay via $\pi^+\to\mu^++\nu_\mu$ and $\mu^+\to e^++\nu_e+\bar\nu_\mu$ as well as charge-conjugated processes. Simultaneously, neutral pions from the same CR interactions decay as $\pi^0\to\gamma+\gamma$ and provide a flux of $\gamma$-rays. The neutrino is therefore a strong probe of multi-messenger sources: it provides unambiguous evidence for the presence of cosmic ray sources and helps to separate hadronic $\gamma$-ray emission from leptonic scenarios.

The ``Multimessenger Astrophysics'' parallel session at NOW 2018 complemented five overview talks in the plenary session that covered recent results of TeV-PeV astrophysical neutrinos ({\bf Claudio Kopper}), recent results of gravitational wave astronomy ({\bf Ian Harry}), neutrino searches from Fermi-LAT $\gamma$-ray blazars ({\bf Matthias Kadler}), charged Galactic cosmic rays observed in orbit ({\bf  Giovanni Ambrosi}), and the prospects of detecting the next core-collapse supernovae via MeV-GeV neutrinos ({\bf Shunsaku Horiuchi}). Details can be found in the corresponding proceedings contributions. In the following, we summarize the twelve presentations given in the parallel session, covering astrophysical neutrinos, supernova neutrinos, $\gamma$-rays, and cosmic rays.

\section{Astrophysical Neutrinos {\it (Markus)}}

{\bf Maurizio Spurio} provided an overview of the ANTARES detector with a focus on recent multi-messenger results. The experiment is located off the shore of Toulon (France) and has been operating since 2007. Its detection volume is only about one percent of the cubic-kilometer IceCube observatory located at the South Pole, but its location in the Northern Hemisphere allows for a compatible sensitivity for sources in the Southern Hemisphere, in particular sources towards the Galactic Center. Maurizio presented limits on point-like sources using ANTARES data collected from 2007 to 2015. ANTARES has been looking for sources in an all-sky search as well as emission from a catalogue of 106 known $\gamma$-ray sources~\cite{Albert:2017ohr}. No significant neutrino emission has been detected, resulting in upper limits for these sources.

The search of diffuse neutrino emission with ANTARES is an important check of recent IceCube observations. The data collected between 2007 and 2015 shows 33 candidate events for an isotropic astrophysical neutrino flux in comparison to $24\pm6$ events expected from backgrounds~\cite{Albert:2017nsd}. This correspond to a mild excess of $1.6\sigma$ and the diffuse flux limit of $E_\nu^2\phi_{\nu_\mu+\bar\nu_\mu}\simeq 4\times10^{-8} {\rm GeV} {\rm s}^{-1} {\rm cm}^{-2} {\rm sr}^{-1}$ is consistent with IceCube's observation. 

The diffusive propagation of Galactic cosmic rays and their interaction with gas in the vicinity of the Galactic Plane should also be visible as diffuse high-energy neutrino emission. A recent joined analysis of ANTARES and IceCube~\cite{Albert:2018vxw} places upper limits on the emission. The result is starting to probe optimistic Galactic diffuse models, that assume a strong dependence of cosmic ray densities with Galactic radius and a high cosmic ray cutoff at 50~PeV~\cite{Gaggero:2015xza}.

An important aspect of neutrino astronomy is the communication with multi-messenger partners. ANTARES participates in the Astrophysical Multimessenger Observatory Network~\cite{Smith:2012eu,AMON} (AMON) as well as the Gamma-ray Coordinate Network~\cite{GCN} (GCN). Well-reconstructed neutrino candidates with an angular resolution of $0.5^\circ$ can generate neutrino alerts in the case of doublet events (0.04 events/yr), events from the direction of local galaxies (10 events/yr), as well as single high-energy events (20 events/yr above 5~TeV and 3-4 events/yr above 30~TeV). 

On the other hand, ANTARES performs offline follow-up studies on transient electro-magnetic sources, ultra-high energy cosmic rays, high-energy neutrinos, and gravitational waves. Recent highlights are the search of neutrinos from fast radio bursts~\cite{Albert:2018euo}, neutrino emission from TXS 0506+056~\cite{Albert:2018kjg}, and the binary neutron star merger GW170817~\cite{ANTARES:2017bia}. Again, no significant neutrino emission above backgrounds were detected for all these candidate sources leading to upper limits on the time-integrated neutrino emission. The telescope will continue observation until the end of 2019 coinciding with the end of the next gravitational wave observation period (O3) of LIGO and VIRGO.

The next-generation neutrino telescope in the Mediterranean will be KM3NeT~\cite{Adrian-Martinez:2016fdl}. {\bf Rosa Coniglione} summarized the present status of the observatory and the expected science potential. The detector will be comprised of two components -- the low-energy detector ORCA off the shore of Toulon (France) and the high-energy detector ARCA  off the shore of Capo Passero (Sicily, Italy). Both detectors use the same detection units (DUs), consisting of vertical slender strings equipped with 18 Digital Optical Modules (DOMs), but with different inter-spring spacing. The individual DOMs use a novel design with an arrangement of 31 3 inch photo-multiplier tubes, that reduces the cost per photocathode area while improving the angular acceptance. For a discussion of the physics with KM3NeT ORCA we refer to the contribution by Dorothea Samtleben.

In its final configuration, KM3NeT ARCA will consist of two building blocks with 115 DUs, each, with an inter-string spacing of 80 meters. The first construction phase (``Phase-1'') is fully funded and consist of 24 DUs for ARCA, corresponding to an effective volume of $\mathcal{O}(0.1)~{\rm km}^{3}$. The first three DUs at the ARCA site were deployed in December 2015 and March 2016. Two of these DUs operated successfully until a junction box failed in March 2017. The recovery of DUs and the restoration of the sea bed infrastructure are ongoing.

The fully instrumented ARCA detector will have excellent sensitivity for point-source and diffuse neutrino searches. The field of view will be complementary to IceCube in the Southern Hemisphere. KM3NeT expects to reconstruct cascade events with an angular resolution better than $2^\circ$ and an energy resolution of better than 5\%. The combined event samples of muon tracks and cascades allow for a discovery of the diffuse neutrino flux observed by IceCube within six months. Optimistic predictions of the Galactic diffuse neutrino emission can be discovered within four years. Galactic supernova remnants, like RXJ1713.7-3946, that are observed in $\gamma$-rays can provide $3\sigma$ evidence for neutrino emission within five years, if the $\gamma$-ray emission is of hadronic nature.

Future upgrades and extensions of the IceCube detector at the South Pole were summarized by {\bf Marek Kowalski}. The collaboration is presently preparing the IceCube-Upgrade, which is an infill of up to seven strings in the existing IceCube DeepCore array. These strings will carry new DOM designs with upgraded electronics, smaller diameter for efficient deployment, larger acceptance and/or effective area. Together with integrated calibration devices (LED flashers, acoustic sensors, and optical cameras), the new strings will also carry stand-alone light sources, that allow for a better calibration of the existing and new strings. The improved knowledge of the glacial ice can be used to reduce systematic uncertainties and the improvement of archival data with respect to angular, energy, and flavor reconstructions. The IceCube-Upgrade will also allow for the study of fundamental neutrino properties, similar to the existing physics program of DeepCore, but with a significantly improved sensitivities (see the contribution by Andrii Terliuk). The deployment of the IceCube-Upgrade strings is foreseen in 2022/23.

The long-term vision for multi-messenger astronomy at the South Pole is IceCube-Gen2~\cite{Aartsen:2014njl} -- a multi-component facility covering low- and high-energy neutrinos. This facility will include an in-ice optical Cherenkov detector with a planned volume of $6-10~{\rm km}^3$, depending on string spacing. In combination with improved calibration methods introduced in IceCube-Upgrade, the IceCube-Gen2 detector will have an improved angular resolution reaching $0.2^\circ$ for horizontal muons. The increased event statistics beyond 100~TeV will allow to probe the spectrum and flavor composition of the diffuse neutrino flux with unprecedented precision. 
%\begin{itemize}
%\item Marek Kowalski - {\it IceCube-Gen2 and IceCube-Upgrade}~\cite{Aartsen:2014njl}
%\item Maurizio Spurio - {\it Overview of ANTARES neutrino telescope: multimessenger results }
%\item Rosa Coniglione - {\it First results and perspectives of the KM3NeT/ARCA Detector}~\cite{Adrian-Martinez:2016fdl}
%\end{itemize}

\section{Supernova Neutrinos {\it (Markus)}}

\begin{itemize}
\item Francesco Capozzi - {\it Distinguishing SN $\nu$ flavour equalization from a pure MSW effect}
\item Rasmus Hansen - {\it Extended Neutrino Sphere Effects on SN $\nu$ Oscillations}
\item Giulia Pagliaroli - {\it Combined search of MeV $\nu$s and GWs from astro sources}
\end{itemize}

\section{Multimessenger studies with gamma rays {\it (Manuela)}}
The Earth atmosphere being opaque to gamma rays, they can either be  detected directly, using space borne experiments, or indirectly, using ground-based experiments that can sample the electromagnetic showers generated by gamma rays. %Two detection techniques can be used for indirect searches, namely the Imaging Atmospheric Cherenkov Telescopes (IACTs)  and the Water Cherenkov technique. 

%adopted in the  

Multi-wavelength and multi-messenger observations are 
essential to understanding the gamma ray sky.

%\item Elisabetta Bissaldi - {\it Multimessenger studies with the Fermi satellite}

{\bf Elisabetta Bissaldi} reports about multimessenger studies with the \textit{Fermi} gamma ray satellite~\cite{Bissaldi}, operating in space since 10 years. 
%The \textit{Fermi} payload is composed by the Gamma Ray Burst Monitor (GBM),   designed to study Gamma Ray Bursts (GRBs) by detecting gamma 
%rays between 8 keV and 40 MeV,  and the  Large Area Telescope (LAT) devoted to the high energies, from 20 MeV to 300 GeV. 
The Fermi satellite is a key instrument to perform multi-messenger astrophysics, giving the high pointing accuracy, as well as the high quality data for both temporal and spectral analyses. The  electromagnetic (EM) counterpart of a gravitational wave generated by the merger of two neutron stars, was unambiguously detected for the first time on August 17th 2017 by the Fermi GBM~\cite{Gold2017}~\cite{Abbott2017}: the event, labelled \textit{GW170817}, was found to take place in the galaxy NGC 4993, located 44 Mpc away. The detected EM emission includes three main components~\cite{TheMMpaper}: (i) a prompt, short GRB, demonstrating the association between GW and GRBs (ii) the first observation in ultraviolet, optical and infrared detection of a  kilonova, an object  due to the radioactive decay of heavy elements formed by neutron capture, followed by (iii) the delayed X-rays and radio counterparts. 
No coincident high-energy neutrino was detected  from the location of \textit{GW170817}. Extended search for neutrinos in the direction of the source was carried out for two weeks following the merger, but it found no significant neutrino emission. 

The search for a TeV gamma ray counterpart of the GW170817 event allowed setting upper limits to the TeV gamma ray emission from the High-Energy Stereoscopic System (H.E.S.S.)
and other IACTs~\cite{TheMMpaper}, as presented by {\bf Andrew Taylor}. 
%HAWC performed the search for TeV gamma rays in association with the binary neutron star coalescence candidate (later designated GW170817) detected by the Virgo and LIGO gravitational wave detectors~\cite{TheMMpaper}. 
{\bf Gerd Kunde} reported that at the time of the trigger the region was not visible to the High Altitude Water Cherenkov detector (HAWC), so that no prompt follow-up was possible. About 9 hours after the trigger, the region became to be visible and was observed for 2 hours, allowing setting 95$\%$ confidence level upper limits for energies above 1 TeV, assuming a -2.5 spectrum. 

A new multimessenger astrophysics milestone was achieved a few weeks later, in September 22nd 2017, when a  neutrino event of about 290 TeV detected by IceCube was found to be coincident in direction and time with a gamma-ray flare from the blazar TXS 0506+056~\cite{IceCube:2018dnn}.
The probability of a high-energy neutrino being detected by chance coincidence with a flaring blazar from Fermi-LAT catalogs was found to be disfavored at a 3$\sigma$ confidence level. Ice Cube archival searches  found a compelling evidence for a three months ``neutrino flare'' in 2014-2015 from the blazar, however no gamma ray  counterpart was found in \textit{Fermi}-LAT data. This event, labelled as \textit{IceCube-170922A}, happened in a period of flaring activity in GeV gamma rays detected by \textit{Fermi}-LAT and 
 above 100 GeV by the Major Atmospheric Gamma-ray Imaging Cherenkov (MAGIC) telescope~\cite{IceCube:2018cha}~\cite{MAGIC}.  The prompt diffusion of the sky position of the IceCube-170922A alert triggered an extensive multiwavelength (MWL) campaign by many telescopes on ground and in space, from radio frequencies up to VHE gamma-rays. 
 
While the detection of the high energy neutrino alone indicates acceleration of protons in the jet of the blazar to energies above the CR \textit{knee} (10$^{15}~eV$), the combination with the EM data allow us to probe the maximum energy that they can attain.

In the VHE band, TXS 0506+056 was followed up by several IACTs, 
%with the earliest one starting a few hours after the neutrino alert. In particular,  1.3 hours of observations in the direction of the blazar TXS 0506+056 were performed using the HESS experiment about 4 hours after the circulation of the neutrino alert. Additional observations on subsequent nights were also carried out, but 
No clear signal was found in HESS, and upper limits at 95$\%$ CL on the gamma-ray flux were derived. 
%$ \Phi < 7.5 \times 10^{-12} cm^{-2} s^{-1} $ during the H.E.S.S. observation period. 
HAWC found no $\gamma$-ray source above 1 TeV at the location of TXS 0506+056, either close to the time of the neutrino alert or in archival data taken since November 2014.

Concerning the study of galactic CR sources with gamma-rays, the recent detection of extended~TeV~gamma-ray emission coincident with the 
locations of two nearby middle-aged pulsars (Geminga and PSR 
B0656+14)~\cite{HAWCpositrons} recently triggered a lot of interest. The HAWC observation of the spectral and spatial properties of these sources are used
to constrain their contribution to the positron flux at Earth, measured by AMS-02~\cite{positrons}. Assuming that all the observed gamma-ray emission at TeV energies is
produced by relativistic electron and positron pairs, the electron and positron
flux produced by these sources at Earth was computed. The measured TeV emission profile is used to constrain the diffusion of particles away from these
sources, and it is found to be much slower than previously assumed. The HAWC Collaboration thus concluded that, under the assumption of isotropic and homogeneous diffusion, the two pulsars are ruled out as sources of the CR positron flux.

{\bf Andrew Taylor} reports about the combined electron plus positron spectrum  measured by the High-Energy Stereoscopic System (H.E.S.S.), located in Namibia, from 250 GeV to 20 TeV. This measurement can be described by a smooth broken power law, with no sharp features. This result allow us to exclude models that describe prominent features from nearby sources such as Vela.

%begin{itemize}
%\item 
%\item Andrew Taylor reports about {\it Multimessenger observations with IACTs}~\cite{Taylor}.





%The production of high-energy neutrinos in astrophysical environments is expected to occur mainly through the decay of charged pions produced in inelastic collisions between high energy protons and ambient target particles, which can be either matter (p-p interactions) or low-energy photons (p-$\gamma$ interactions). The present multimessenger studies are able to constrain this scenarios and shed light on the neutrino and cosmic ray origin. 


%The High Altitude Water Cherenkov detector is
%located in Mexico at an altitude of 4100 meters and
%is used to sample the charged particles created in the air showers generated by TeV gamma rays passing through the Earth atmosphere. It is sensitive to the energy range 0.5-100 TeV and
%during each 24 hour period it observes two-thirds of the sky. 








%Gamma rays at these energies are mainly produced by 
%inverse Compton scattering of  positrons or electrons on
%lower-energy photons (CMB) as well as by by synchrotron radiation. 
%The diffusion coefficient increases
% is chosen to be 1/3, motivated by the Kolmogorov turbulence
%model.





%\end{itemize}

\section{Cosmic Rays from galactic and extragalactic sources {\it (Manuela)}}
Little intro, as it was done before.\\
{\bf Fiorenza Donato} provides a summary about recent improvements in the field of galactic cosmic rays (CRs). Following the hints provided by previous balloon experiments, indicating a transition in the spectral index of CR 
proton, helium and heavier nuclei, the 
PAMELA Collaboration published in 2011
precise measurements of proton and helium fluxes as a function of rigidity (the rigidity is given by the particle momentum over the charge $R=\frac{pc}{Ze}$)~\cite{PAMELApHe} between 1 GV to 1.2 TV, showing a clear feature above 200 GeV. 
The AMS-02 collaboration in 2015~\cite{proton}~\cite{Aguilar:2015ctt} showed  
that both the proton and helium spectra
 cannot be described as a single power law (between 1 GV to 1.8 TV), and that
a transition in the spectral index takes place above 200 GeV.  Moreover, not only the helium flux cannot be described by a single power law, but the ratio
between the proton and the helium flux is rigidity-dependent. This behavior is not expected, and it was also observed in the flux ratio of other species,
like C/p, O/p~\cite{heco}. 

Lithium, Beryllium and Boron are mostly secondary CR species,
produced by fragmentation of heavy nuclei (mainly C N O) propagating in the interstellar medium, made of protons and helium. 
The three fluxes~\cite{libeb} deviate from a single power law above 200 GV in an identical way.
The  secondary to primary flux ratios are very sensitive to the propagation of CRs in the Milky Way. In particular, the Boron to Carbon flux ratio (hereafter B/C), has been measured by a variety of CR experiments and
can be considered as \textit{standard candle} for the study of galactic CRs propagation. 

In order to reproduce the observed transition in the spectral index above 200 GeV, the transition can be generated  either at the source (injection or acceleration) 
either during the propagation in the Milky Way. The first indications in favor of a diffusive propagation origin for these spectral features was provided in~\cite{Genolini:2017dfb}, based on a semi-analytic approach for the transport equation. 

%The observed hardening could be at least partially explained by a re-acceleration inside the sources of CRs, as discussed in~\cite{Blasi:2017caw}.
It is also important to mention another unexpected result: the fact that positrons, protons and antiprotons show identical rigidity dependence above 60 GV, while electrons have a different behavior~\cite{antip}. This results can be interpreted as a  coincidence or as the actual result of underlying physics. In the last case it would imply that both antiprotons and positrons are secondaries, implying small energy losses for these particles, in striking disagreement with the conventional scenario in which positrons undergo severe energy losses during their propagation, in view of the low mass (about 1/2000 times smaller than antiprotons).

Two important points that emerge in view of the precision of AMS-02 data: the comparison between the expected antiproton flux from conventional astrophysical processes (whose \textit{baseline} is derived from the B/C analysis) and the precise AMS-02 data is still an open issue, especially because the antiproton production cross sections should be known with high accuracy in order to avoid introducing high theoretical uncertainties. 
In this context, the recent  LHCb data improve the experimental scenario. In view of the previous points, antiprotons data allow setting strong upper bounds on the dark matter annihilation cross section or to improve the fit with respect to the secondaries 
alone adding a tiny DM contribution. In other words, the possible room left for a primary component, likely due to dark matter (DM), is not yet clearly established, even though recent papers claim for a statistically significant excess with respect to the astrophysical background. 

CR electrons and positrons account for about $1\%$ of CRs. However, they carry important information about the
local universe, since the typical propagation length of TeV electrons and positrons is smaller than $\simeq 0.3 kpc$. 

Recent positrons measurements are not in agreement with the astrophysical background from conventional models, leaving the room for a primary component whose nature, either astrophysical or exotic, is still highly debated. At present, it cannot be excluded that nearby and
very old sources may contribute to the flux at Earth even if they are no longer visible at any
energy of the electromagnetic band. A strategy to constrain the source contributions of local
known sources is to model their multi-wavelength emission and to connect it to the emitted
CRs. A recent work~\cite{Manconi:2018azw} aims at  quantifying the contribution of local known
sources, in particular from two SNRs which are widely considered as the main candidates, namely Vela and Cygnus Loop, by combining three complementary measurements: (i) the radio flux from Vela and Cygnus Loop at all the available frequencies, (ii) the combined electron plus positron flux from five experiments from the GeV to tens of TeV energy, (iii) the electron plus positron dipole anisotropy upper limits from 50 GeV to about 1 TeV. 
The model is compatible with all the measurements, including the anisotropy bounds, providing strong arguments towards the understanding of the sources of CR electrons and positrons. 


{\bf Martin Lemoine} summarizes the current understanding of UHECRs origin, in view of the most recent experimental findings, while the Anisotropy Studies with the Pierre Auger Observatory are presented by {\bf Rogerio de Almeida}.

The two largest and most precise detectors of UHECR currently in operation are the
Pierre Auger Observatory (PAO) in Argentina (Mendoza) and the Telescope Array in the
USA (Utah). Both detectors exploit the hybrid concept, combining an array of
surface detectors to sample extensive air showers when they reach the ground and
fluorescence telescopes, overlooking the surface array, to collect the fluorescence light of the excited atmospheric nitrogen. 

The main challenge in this field is currently to understand the observed UHECR spectra and mass composition and relating them to potential sources, most likely extra-galactic~\cite{Aloisio}. 
The PAO energy spectrum of UHECRs shows a clear suppression at $E_{s} = (3.9 \pm 0.2(stat.) \pm
0.8(syst.)) \times 10^{19} eV$, compatible with the expected value for the GZK cutoff, where the universe becomes opaque to protons. However, this spectral feature could also be indicating the maximal energy that can be achieved at acceleration sites. 
The UHECR composition and arrival direction constitute two additional observables to shed light on the problem. The PAO reports evidence for a mixed composition at high energies, resulting in an increasingly heavier composition above $10^{18} eV$. The results from the Telescope Array (TA), initially in tension with those from PAO, have been shown to be compatible with mixed composition which best describes the PAO data~\cite{vitor}. 

The PAO and TA, being located on two different hemispheres, have access to different regions of the sky: the TA reported the existence of a hot spot in the Northern sky with a significance of $3.4~\sigma$, while 
%the PAO reports 
%a strong evidence (5.2$\sigma$) of dipolar anisotropy for CRs with energies above 8 EeV~\cite{Aab:2017tyv}.
a recent analysis by the PAO~\cite{Aab:2018chp} provides evidence in the arrival direction of
UHECRs on an intermediate angular scale, indicative of excess arrivals from strong nearby sources. 
The arrival directions of 5514 events with energies above 39 EeV are compared
with two distinct populations of extra-galactic $\gamma$-ray emitters: active galactic nuclei from the second catalog of hard Fermi-LAT sources (2FHL) and starburst galaxies from a sample that was examined with \textit{Fermi}-LAT. The starbust model fits the data better than the hypothesis of isotropy, with statistical significance of $4\sigma$. However, caution is required when identifying these objects as the preferred 
sources prior to understanding the impact of magnetic deflections. 

Large scale anisotropies can be used to infer aspects of the global distribution of sources. A strong evidence (5.2$\sigma$) of dipolar anisotropy for CRs with energies above 8 EeV is reported~\cite{Aab:2017tyv}, while no significant second harmonic amplitude is observed. This results can be interpreted as a strong evidence against the galactic origin of UHECRs. The energy dependence of the amplitude and direction of the three-dimensional dipole are presented in~\cite{Aab:2018mmi} for events with energy higher than 4 EeV. The energy dependence is expected, because of the smaller deflection experienced
by UHECRs at higher rigidities as well as large attenuation suffered by UHECRs from distant sources.  No clear dependence on energy as found for the dipole direction, but regardless the energy, the dipole direction point towards a region outside our galaxy. 
Plausible explanations for the observed dipole-like structure
include the effect of the diffusive propagation as 
as well as the effect of the known anisotropy in the distribution of the sources. 



The previously discussed experimental findings must be combined in order to shed light on the origin of UHECRs. The possibility to point back to the sources of UHECRs depends crucially on the composition, since  
at a given energy $E$, a nucleus of charge $Z$ is typically deflected by an angle $Z$ times larger than a proton, so that protons are expected to produce stronger anisotropies than nuclei. 
Moreover, secondary gamma rays and neutrinos~\cite{AlvesBatista} are generated in 
interactions whose energy threshold depends on the mass number $A$, implying that the flux of these particles also depend on the UHECR composition. In this frame, the pure proton composition is in tension with secondary diffuse backgrounds of TeV gamma rays and neutrinos. 

\section{Conclusions {\it (Manuela \& Markus)}}


\section*{Acknowledgements}
MA acknowledges support by Danmarks Grundforskningsfond (project no.~1041811001) and \textsc{Villum Fonden} (project no.~18994).

\bibliographystyle{JHEP}
\bibliography{references}


\end{document}
